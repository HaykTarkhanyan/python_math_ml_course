 \begin{center}\begin{large} Homework Problems 4
 \end{large}\end{center}
 \bigskip


\begin{problem}%[2 points]
    Find the eigenvalues and eigenvectors of the following matrices:

    \begin{enumerate}
        \item[a) ] $A=\begin{bmatrix}
            1&2\\0&1\end{bmatrix}$,

        \item[b) ] $B=\begin{bmatrix}
            2&1\\1&2
        \end{bmatrix}$,
        
        \item[c) ] $C=\begin{bmatrix}
            6&6&-12\\4&2&-6\\4&3&-7
        \end{bmatrix}$,
        
        \item[d) ] $D=\begin{bmatrix}
-1&0&0\\0&-1&0\\0&0&3        \end{bmatrix}$.
    \end{enumerate}

    % What are the algebraic and geometric multiplicities of the found eigenvalues?
\end{problem}

% \begin{problem}
%     Check if the following matrices are positive definite:

%     \begin{enumerate}
%         \item[a) ] $A=\begin{bmatrix}
%             10&5&2\\5&3&2\\2&2&3
%         \end{bmatrix}$,

%         \item[b) ] $B=\begin{bmatrix}
%             1&4&10\\-1&2&0\\7&2&1
%         \end{bmatrix}$.
%     \end{enumerate}
% \end{problem}

% \bigskip

% \begin{problem}
%    Find, if exists, the limit of the sequence, as $n\to\infty$:

%     \begin{enumerate}
%         \item[a) ] $\dfrac{1}{n^2}$,
        
%         \item[b) ] $\dfrac{3}{2n}$,
        
%         \item[c) ] $\dfrac{n^2}{2-n^3}$,
        
%         \item[d) ] $\dfrac{(n-1)(2-n^3)}{n^7}$,
        

%         \item[e) ] $(0.9)^n$,

%         \item[f) ] $2^n$,

%         \item[g) ] $\sin {\pi n}$,

%         \item[h) ] $\dfrac{\sin {\pi n}}{n^2}$.

%     \end{enumerate}
%     Which of the sequences above are convergent?

% \end{problem}
% \bigskip
% \newpage
\begin{problem}
    Find, if exist, the points at which the function is discontinuous:

    \begin{enumerate}
     \item[a) ] $f(x) = \begin{cases} x - 2, &  x \leq 0 \\ 2, &  x > 0 \end{cases}$
\\
     \smallskip
     \includesvg[ width=0.31\textwidth, keepaspectratio]{figs/2a}
     \smallskip
        
     \item[b) ] $f(x) = \begin{cases} x + 2, &  x > 0  \\ -\frac{1}{2}x^{2}+2, &  x \le 0 \end{cases}$
  \\
  \smallskip
     \includesvg[ width=0.31\textwidth, keepaspectratio]{figs/2b}
     \smallskip
        
        
        \item[c) ] $f(x) = \begin{cases} x + 2, &  x > 0 \\ -5, &  x = 0 \\ -\frac{1}{2}x^{2}+2, &  x < 0 \end{cases}$
  \\
  \smallskip
     \includesvg[ width=0.31\textwidth, keepaspectratio]{figs/2c}
     % \newpage
        
        
  %       \item[d) ] $f(x) = \begin{cases} \ln(x+2), &  x \leq 1, x \neq -1 \\ -1, &  x = -1 \\ -1, &  x > 1 \end{cases}$
  % \\
  % \smallskip
  %    \includesvg[ width=0.31\textwidth, keepaspectratio]{figs/2d}
        
        
    \end{enumerate}
\end{problem}
\bigskip
\begin{problem}
    Write the Taylor polynomial for the function $e^{x^2}$, around the point $x=0$.
\end{problem}
% \begin{problem}
%     For which values of $x$ is the function continuous?

%     \begin{enumerate}
%         \item[a) ] $f(x) = \dfrac{2}{x^{2} + 1}$,
        
%         \item[b) ] $g(x) = \dfrac{2x+1}{x^{2}+x-2}$,
        
%         \item[c) ] $h(x) = \begin{cases} \dfrac{x^{2} - 1}{x-1}, &  x\neq 1\\
% 2, &  x = 1 \end{cases}$
%         \end{enumerate}
% \end{problem}
% \bigskip

% TRKH-info: maybe use this for gorcnakan

% \begin{problem}
% Find the values of  
% $a$
%   that make the function  
% $f
% (
% x
% )
% $  continuous for all real numbers:

% \begin{equation*}
% f(x) = \left\{ \begin{array}{rl} 4x+5, &  \mbox{if \(x\geq -2\),} \\ x^2+a, &  \mbox{if \(x < -2\).} \end{array} \right.
% \end{equation*}

% \end{problem}

\bigskip

\begin{problem}
Calculate $f'(x)$:
    \begin{enumerate}
    \item[a) ] $10000$,

        \item[b) ] $2x^2-7x+1$,
        
        \item[c) ] $2\sin x \cdot e^x$,
        
        \item[d) ] $x^2(2-x)$,
        % \newpage
        \item[e) ] $e^{3x^2}$,
        \item[f) ] $\dfrac{2x-3}{4-x^3}$,
        \item[g) ] $\ln(\ln x)$.
    \end{enumerate}
\end{problem}

\bigskip

\begin{problem}
Find the extrema of $f(x)$ on the given set:
    \begin{enumerate}
        \item[a) ] $f(x)=3-|x+4|,\: x\in\R$,
        \item[b) ] $f(x)=2x^3+18x^2+54x+50,\: x\in\R$,
        \item[c) ] $f(x) =
      \displaystyle{\frac{2x}{x^2-2x+2}},\: x\in[0,3]$,
        \item[d) ] $f(x)=\dfrac{x^2}{x-1},\: $ on its domain,
        \item[e) ] $f(x)=\sin(x^2),\: $ on its domain.
    \end{enumerate}
\end{problem}

% \bigskip
% \bigskip


% \begin{center}
%     \begin{large}
%         Solutions
%     \end{large}
% \end{center}


% \bigskip
% \begin{solution}
%     a) Calculating the determinants of all upper-left square submatrices of $A$, we have:
%     \[ \Delta_1 = 10>0, \]
%     \[ \Delta_2 = \left|\begin{array}{cc}
%          10&5  \\
%          5&3 
%     \end{array}\right|=30-25=5>0, \]
%     \[ \Delta_3 = \left|\begin{array}{ccc}
%          10&5 &2 \\
%          5&3 &2\\
%          2&2&3
%     \end{array}\right|=3>0, \]
%     therefore $A$ is positive definite: $A \succ 0$.

%     b) $B$ is not symmetric, hence it is not positive definite (neither negative or semi-definite).
% \end{solution}

% \bigskip
% \begin{solution}
% \begin{enumerate}
%     \item[a) ] 0,
%     \item[b) ] 0,
%     \item[c) ] $\lim\limits_{n\to\infty} \dfrac{1/n}{2/n^3-1} = \dfrac{0}{-1}=0$,
%     \item[d) ] 0,
%     \item[e) ] 0,
%     \item[f) ] $+\infty$,
%     \item[g) ] $\sin \pi n = 0$ for any $n\in\N$,
%     \item[h) ] 0.
% \end{enumerate}
% \end{solution}
% \smallskip


% \begin{solution}
% \begin{enumerate}
%     \item[a) ] 0,
%     \item[b) ] $f(x)$ is continuous everywhere,
%     \item[c) ] 0,
%     \item[d) ] -1 and 1.
% \end{enumerate}
% \end{solution}
% \smallskip


% \begin{solution}
% \begin{enumerate}
%     \item[a) ] $\R$ (i.e. for any value of $a$),
%     \item[b) ] $\R\setminus\{-2,1\}$,
%     \item[c) ] $\R$ (because $\dfrac{x^2-1}{x-1}=x+1$.
% \end{enumerate}
% \end{solution}
% \smallskip


% \begin{solution}
% \[ \lim\limits_{x\to -2} 4x+5 = \lim\limits_{x\to -2} x^2+a \]
% \[ -3= 4+a \]
% \[ a= 7 \]
% \end{solution}
% \smallskip


% \begin{solution}
% \begin{enumerate}
%     \item[b) ] $12x^3+\dfrac{1}{x^2}$,
%     \item[d) ] $e^x+xe^x$,
%     \item[e) ] $6xe^{3x^2}$,
%     \item[g) ] $\dfrac{1}{\ln x}\cdot \dfrac{1}{x}$.
% \end{enumerate}
% \end{solution}
% \smallskip


% \begin{solution}
% \begin{enumerate}
%     \item[d) ] $f(x)=\dfrac{x^2}{x-1}$

%     The domain of this function is $(-\infty,1)\cup(1,+\infty)$ (in other words, it's \textit{not} continuous at $x=1$), so we must look into each of those intervals \textit{separately}.

%     As $f(x)$ is continuous on $(-\infty,1)$, we can proceed finding its critical points:
%     \[  f'(x)=\dfrac{2x(x-1)-x^2}{(x-1)^2}=\dfrac{x^2-2x}{(x-1)^2} =0\]
%     The critical points are $x=0$ (the other root $x=2$ doesn't belong to the interval). Investigating the sign of $f'(x)$ between those points, we can find out the behavior (ascending/descending) of $f(x)$ in each interval:
%     \medskip
%   \begin{center}
% \begin{tikzpicture}
%   % Axis
%   \draw[-] (-2,0) -- (1,0) node[right] {};
%   \draw[dashed,->] (1,0) -- (4,0) node[right] {$x$};
%   % \draw (0,-0.2) -- (0,0.2); % y-axis
  
%   % Points
%   \foreach \x/\label in {-1/-1, 0/0, 1/1}
%     \filldraw (\x,0) circle (2pt) node[below] {$\label$};
  
%   % Arrow
%   \draw[->, thick] (-1,0.3) -- (-0.1,0.8) node[midway, above] {};
%   \draw[->, thick] (0.1,0.8) -- (0.9,0.3) node[midway, above] {};
%   % \draw[->, thick] (5.1,0.3) -- (7.9,1.3) node[midway, above] {};
%   % \draw[->, thick] (8.1,1.3) -- (7.9,1.3) node[midway, above];
% \end{tikzpicture}
%   \end{center}
%   \medskip
% Therefore, $x=0$ is a local maximum point.

% Analogously, $x=2$ is a local minimum point.

% Another possible approach would be finding the sign of $f''(x)$ at the points $x=0$ and $x=2$.


%     \item[e) ] $f(x)=\sin(x^2)$
%     \[  f'(x)=2x\cos(x^2) =0 \]
%     The critical points are $x=0$ and all real numbers of the form $\sqrt{\frac{\pi}{2}+\pi k}$ (where $\cos(x^2)=0$), $k=1,2,\dots$. Investigating the sign of $f'(x)$ between those points, we can find out the behavior (ascending/descending) of $f(x)$ in each interval:
%     \medskip
%   \begin{center}
% \begin{tikzpicture}
%   % Axis
%   \draw[->] (-1,0) -- (9,0) node[right] {$x$};
%   % \draw (0,-0.2) -- (0,0.2); % y-axis
  
%   % Points
%   \foreach \x/\label in {0/0, 2/\dfrac{\pi}{2},5/\dfrac{\pi}{2}+\pi,8/\dfrac{\pi}{2}+2\pi}
%     \filldraw (\x,0) circle (2pt) node[below] {$\label$};
  
%   % Arrow
%   \draw[->, thick] (0,0.3) -- (1.9,1.3) node[midway, above] {};
%   \draw[->, thick] (2.1,1.3) -- (4.9,0.3) node[midway, above] {};
%   \draw[->, thick] (5.1,0.3) -- (7.9,1.3) node[midway, above] {};
%   % \draw[->, thick] (8.1,1.3) -- (7.9,1.3) node[midway, above];
% \end{tikzpicture}
%   \end{center}
%   \medskip
% Therefore, $x=0$ and $x=\dfrac{\pi}{2}+\pi+2\pi k$, $k=0,1,2,\dots$ are local minimum points, while $x=\dfrac{\pi}{2}+2\pi k$ are local maximum points.
% \end{enumerate}
% \end{solution}
% \smallskip

