\documentclass[10pt,a4paper]{article}
\usepackage[utf8]{inputenc}
\usepackage[T1]{fontenc}
\usepackage{amsmath,amssymb,amsfonts}
\usepackage{geometry}
\usepackage{multicol}
\usepackage{enumitem}
\usepackage{booktabs}
\usepackage{array}
\usepackage{xcolor}
\usepackage{tcolorbox}
\usepackage{fancyhdr}
\usepackage{titlesec}
\usepackage{hyperref}

% Page geometry - narrow margins for cheat sheet
\geometry{left=1.5cm, right=1.5cm, top=1.5cm, bottom=1.5cm}

% Colors
\definecolor{sectionblue}{RGB}{44, 82, 130}
\definecolor{subsectiongray}{RGB}{45, 55, 72}
\definecolor{formulabg}{RGB}{247, 250, 252}
\definecolor{tipgreen}{RGB}{56, 161, 105}
\definecolor{boxblue}{RGB}{235, 248, 255}

% Section formatting
\titleformat{\section}{\large\bfseries\color{sectionblue}}{\thesection.}{0.5em}{}[\titlerule]
\titleformat{\subsection}{\normalsize\bfseries\color{subsectiongray}}{\thesubsection}{0.5em}{}
\titlespacing*{\section}{0pt}{12pt}{6pt}
\titlespacing*{\subsection}{0pt}{8pt}{4pt}

% Custom box for formulas
\newtcolorbox{formulabox}{
    colback=formulabg,
    colframe=gray!50,
    boxrule=0.5pt,
    left=4pt, right=4pt, top=2pt, bottom=2pt
}

% Custom box for tips
\newtcolorbox{tipbox}{
    colback=green!5,
    colframe=tipgreen,
    boxrule=0.5pt,
    left=4pt, right=4pt, top=2pt, bottom=2pt,
    fontupper=\small\itshape
}

% Compact lists
\setlist{noitemsep, topsep=2pt, parsep=2pt, partopsep=0pt}

% Header
\pagestyle{fancy}
\fancyhf{}
\fancyhead[L]{\small\textbf{Mathematics 1 - Cheat Sheet}}
\fancyhead[R]{\small Page \thepage}
\renewcommand{\headrulewidth}{0.4pt}

\begin{document}

\begin{center}
{\LARGE\bfseries\color{sectionblue} Mathematics 1 -- Comprehensive Cheat Sheet}
\end{center}

\vspace{-0.3cm}

%==============================================================================
\section{Summation and Product Notation}
%==============================================================================

\textbf{Definitions:}
\[
\sum_{i=m}^{n} a_i = a_m + a_{m+1} + \cdots + a_n \qquad\qquad
\prod_{i=m}^{n} a_i = a_m \cdot a_{m+1} \cdot \cdots \cdot a_n
\]

\textbf{Important Sums:}
\begin{center}
\begin{tabular}{ll}
\toprule
\textbf{Name} & \textbf{Formula} \\
\midrule
Arithmetic & $\displaystyle\sum_{i=1}^{n} i = \frac{n(n+1)}{2}$ \\[8pt]
Sum of squares & $\displaystyle\sum_{i=1}^{n} i^2 = \frac{n(n+1)(2n+1)}{6}$ \\[8pt]
Geometric & $\displaystyle\sum_{i=0}^{n} r^i = \frac{1-r^{n+1}}{1-r}$ for $r \neq 1$ \\[8pt]
Constant & $\displaystyle\sum_{i=1}^{n} c = n \cdot c$ \\
\bottomrule
\end{tabular}
\end{center}

\textbf{Properties:} $\sum(a_i + b_i) = \sum a_i + \sum b_i$; \quad $\sum c \cdot a_i = c \cdot \sum a_i$; \quad $\prod(a_i \cdot b_i) = \prod a_i \cdot \prod b_i$

\begin{tipbox}
\textbf{Template:} To convert to summation notation: (1) Identify the pattern, (2) Write general term $a_i$ as function of index, (3) Determine start/end indices.
\end{tipbox}

%==============================================================================
\section{Functions and Graph Transformations}
%==============================================================================

\begin{center}
\begin{tabular}{ll|ll}
\toprule
\textbf{Transform} & \textbf{Effect} & \textbf{Transform} & \textbf{Effect} \\
\midrule
$f(x) + c$ & Shift UP by $c$ & $c \cdot f(x)$ & Vertical stretch by $c$ \\
$f(x) - c$ & Shift DOWN by $c$ & $f(c \cdot x)$ & Horizontal compress by $c$ \\
$f(x + c)$ & Shift LEFT by $c$ & $-f(x)$ & Reflect across $x$-axis \\
$f(x - c)$ & Shift RIGHT by $c$ & $f(-x)$ & Reflect across $y$-axis \\
\bottomrule
\end{tabular}
\end{center}

\textbf{Key Concepts:} Domain $D_f$ = valid inputs; Image/Range $W_f$ = actual outputs; Codomain = all possible outputs.

%==============================================================================
\section{Exponential and Logarithmic Functions}
%==============================================================================

\subsection{Exponential Rules}
\[
e^{a+b} = e^a \cdot e^b \qquad e^{a-b} = \frac{e^a}{e^b} \qquad (e^a)^b = e^{ab} \qquad e^0 = 1 \qquad e^{\ln x} = x
\]

\subsection{Logarithm Rules}
\[
\ln(ab) = \ln a + \ln b \qquad \ln\left(\frac{a}{b}\right) = \ln a - \ln b \qquad \ln(a^n) = n\ln a
\]
\[
\ln 1 = 0 \qquad \ln e = 1 \qquad \ln(e^x) = x
\]

\textbf{Change of Base:} $\displaystyle\log_a x = \frac{\ln x}{\ln a} = \frac{\log_b x}{\log_b a}$

\textbf{Special:} $\log_a a = 1$; \quad $\log_a 1 = 0$; \quad $\log_a a^n = n$; \quad $a^{\log_a x} = x$

\begin{tipbox}
\textbf{Solving:} For $e^{f(x)} = c$: take ln to get $f(x) = \ln c$. For $\ln(f(x)) = c$: exponentiate to get $f(x) = e^c$. Always check domain!
\end{tipbox}

%==============================================================================
\section{Polynomial Long Division}
%==============================================================================

\textbf{Algorithm:} (1) Arrange in descending powers, (2) Divide leading terms, (3) Multiply and subtract, (4) Repeat until $\deg(\text{remainder}) < \deg(\text{divisor})$.

\textbf{Result:} $P(x) = Q(x) \cdot D(x) + R(x)$

\textbf{Applications:} Finding roots of polynomials, simplifying rational functions, partial fraction decomposition.

%==============================================================================
\section{Inverse Functions}
%==============================================================================

\textbf{Definition:} $f^{-1}$ is the inverse of $f$ iff: $f(f^{-1}(x)) = x$ and $f^{-1}(f(x)) = x$.

\textbf{Existence:} Function must be \textit{bijective} (one-to-one AND onto). For continuous functions, \textit{strictly monotonic} $\Rightarrow$ invertible.

\textbf{Domain/Range swap:} $\text{Dom}(f^{-1}) = \text{Range}(f)$; \quad $\text{Range}(f^{-1}) = \text{Dom}(f)$.

\textbf{Graphically:} Graph of $f^{-1}$ is reflection of $f$ across line $y = x$.

\begin{tipbox}
\textbf{Template:} (1) Write $y = f(x)$, (2) Swap $x$ and $y$, (3) Solve for $y$, (4) Write $f^{-1}(x) = y$, (5) Determine domain.
\end{tipbox}

%==============================================================================
\section{Derivatives -- Basic Rules}
%==============================================================================

\begin{center}
\begin{tabular}{ll|ll}
\toprule
\textbf{Function} & \textbf{Derivative} & \textbf{Function} & \textbf{Derivative} \\
\midrule
$c$ (constant) & $0$ & $\sin x$ & $\cos x$ \\
$x$ & $1$ & $\cos x$ & $-\sin x$ \\
$x^n$ & $nx^{n-1}$ & $\ln x$ & $1/x$ \\
$e^x$ & $e^x$ & $\log_a x$ & $\frac{1}{x \ln a}$ \\
$a^x$ & $a^x \ln a$ & $|x|$ & $\text{sgn}(x) = \frac{x}{|x|}$ \\
\bottomrule
\end{tabular}
\end{center}

\subsection{Differentiation Rules}
\begin{center}
\begin{tabular}{ll}
\toprule
\textbf{Rule} & \textbf{Formula} \\
\midrule
Sum/Difference & $(f \pm g)' = f' \pm g'$ \\
Constant Multiple & $(cf)' = cf'$ \\
Product Rule & $(fg)' = f'g + fg'$ \\
Quotient Rule & $\left(\frac{f}{g}\right)' = \frac{f'g - fg'}{g^2}$ \\
Chain Rule & $(f(g(x)))' = f'(g(x)) \cdot g'(x)$ \\
\bottomrule
\end{tabular}
\end{center}

\textbf{Newton Quotient:} $\displaystyle f'(x) = \lim_{h \to 0} \frac{f(x+h) - f(x)}{h}$

%==============================================================================
\section{Chain Rule and Composite Functions}
%==============================================================================

\textbf{Chain Rule:} $\displaystyle\frac{d}{dx}[f(g(x))] = f'(g(x)) \cdot g'(x)$ \quad or \quad $\displaystyle\frac{dy}{dx} = \frac{dy}{du} \cdot \frac{du}{dx}$

\subsection{Common Patterns}
\begin{center}
\begin{tabular}{ll}
\toprule
\textbf{Function} & \textbf{Derivative} \\
\midrule
$e^{g(x)}$ & $e^{g(x)} \cdot g'(x)$ \\
$\ln(g(x))$ & $\frac{g'(x)}{g(x)}$ \\
$[g(x)]^n$ & $n[g(x)]^{n-1} \cdot g'(x)$ \\
$f(ax+b)$ & $a \cdot f'(ax+b)$ \\
$[g(x)]^{h(x)}$ & $[g(x)]^{h(x)}\left[h'(x)\ln g(x) + h(x)\frac{g'(x)}{g(x)}\right]$ \\
\bottomrule
\end{tabular}
\end{center}

\subsection{Maximum Location Under Transformations}
If $f(x)$ has maximum at $x^*$:
\begin{center}
\begin{tabular}{ll}
\toprule
\textbf{New Function} & \textbf{Max Location} \\
\midrule
$f(x) + c$ & $x^*$ (unchanged) \\
$f(x+c)$ & $x^* - c$ \\
$c \cdot f(x)$, $c > 0$ & $x^*$ (unchanged) \\
$f(cx)$, $c > 0$ & $x^*/c$ \\
$h(f(x))$, $h$ increasing & $x^*$ (unchanged) \\
$f(h(x))$, $h$ increasing & $h^{-1}(x^*)$ \\
\bottomrule
\end{tabular}
\end{center}

%==============================================================================
\section{Convexity and Concavity}
%==============================================================================

\textbf{Convex (concave up):} $f''(x) > 0$ -- ``holds water'', ``smiling''

\textbf{Concave (concave down):} $f''(x) < 0$ -- ``sheds water'', ``frowning''

\textbf{Inflection point:} Where $f''(x) = 0$ AND concavity changes.

\textbf{Formal Definition:} $f$ is convex on $I$ if for all $x_1, x_2 \in I$ and $\lambda \in [0,1]$:
\[
f(\lambda x_1 + (1-\lambda)x_2) \leq \lambda f(x_1) + (1-\lambda)f(x_2)
\]

\begin{tipbox}
\textbf{Key:} Convex functions: local min = global min. Concave functions: local max = global max.
\end{tipbox}

%==============================================================================
\section{Sequences and Series}
%==============================================================================

\begin{center}
\begin{tabular}{lll}
\toprule
\textbf{Type} & \textbf{Closed Form} & \textbf{Recursive} \\
\midrule
Arithmetic & $s_n = a + (n-1)d$ & $s_{n+1} = s_n + d$ \\
Geometric & $s_n = ar^{n-1}$ & $s_{n+1} = r \cdot s_n$ \\
\bottomrule
\end{tabular}
\end{center}

\textbf{Series Formulas:}
\begin{itemize}
    \item Arithmetic (finite): $\displaystyle\sum_{i=1}^{n} a_i = \frac{n(a_1 + a_n)}{2} = \frac{n}{2}(2a + (n-1)d)$
    \item Geometric (finite): $\displaystyle\sum_{i=0}^{n} ar^i = a \cdot \frac{1-r^{n+1}}{1-r}$
    \item Geometric (infinite, $|r| < 1$): $\displaystyle\sum_{i=0}^{\infty} ar^i = \frac{a}{1-r}$
\end{itemize}

%==============================================================================
\section{Mathematical Induction}
%==============================================================================

\textbf{Step 1 (Base Case):} Prove $P(n_0)$ is true.

\textbf{Step 2 (Inductive Step):} Assume $P(k)$ is true (inductive hypothesis). Then prove $P(k+1)$ is true.

\textbf{Conclusion:} By induction, $P(n)$ holds for all $n \geq n_0$.

%==============================================================================
\section{L'H\^opital's Rule and Limits}
%==============================================================================

\textbf{L'H\^opital's Rule:} If $\displaystyle\lim_{x \to a} \frac{f(x)}{g(x)}$ gives $\frac{0}{0}$ or $\frac{\pm\infty}{\pm\infty}$, then:
\[
\lim_{x \to a} \frac{f(x)}{g(x)} = \lim_{x \to a} \frac{f'(x)}{g'(x)}
\]
(provided the right side exists)

\subsection{Indeterminate Forms}
\begin{center}
\begin{tabular}{ll}
\toprule
\textbf{Form} & \textbf{Strategy} \\
\midrule
$\frac{0}{0}$ or $\frac{\infty}{\infty}$ & L'H\^opital directly \\
$0 \cdot \infty$ & Rewrite as $\frac{0}{1/\infty}$ or $\frac{\infty}{1/0}$ \\
$\infty - \infty$ & Combine fractions or factor \\
$1^\infty$, $0^0$, $\infty^0$ & Take logarithm first \\
\bottomrule
\end{tabular}
\end{center}

\textbf{Important Limits:}
\[
\lim_{x \to 0} \frac{\sin x}{x} = 1 \qquad \lim_{x \to 0} \frac{e^x - 1}{x} = 1 \qquad \lim_{x \to 0} \frac{\ln(1+x)}{x} = 1
\]
\[
\lim_{x \to \infty} \frac{e^x}{x^n} = \infty \qquad \lim_{x \to \infty} \frac{x^n}{e^x} = 0 \qquad \lim_{x \to \infty} x^n e^{-x} = 0
\]

%==============================================================================
\section{Curve Sketching -- Systematic Approach}
%==============================================================================

\begin{center}
\begin{tabular}{lll}
\toprule
\textbf{Step} & \textbf{Find} & \textbf{Method} \\
\midrule
1. Domain & Where $f(x)$ defined & Check denominators, roots, logs \\
2. Intercepts & $y$: $f(0)$; $x$: $f(x)=0$ & Direct evaluation \\
3. Symmetry & Even/Odd & $f(-x) = ?$ \\
4. Asymptotes & Vertical, Horizontal, Oblique & Limits at boundaries \\
5. $f'(x)$ & Critical pts, inc/dec & $f'=0$, sign analysis \\
6. $f''(x)$ & Inflection pts, convexity & $f''=0$, sign analysis \\
7. Extrema & Max/min classification & $f'=0$ with $f''$ test \\
\bottomrule
\end{tabular}
\end{center}

\textbf{Example -- Logistic Function:} $S(t) = \frac{1}{1 + e^{-\mu t}}$ with $\mu > 0$

Domain: all $\mathbb{R}$; Limits: $\lim_{t \to -\infty} S(t) = 0$, $\lim_{t \to \infty} S(t) = 1$; Always increasing (S-shaped); Inflection at $t = 0$.

%==============================================================================
\section{Taylor Expansion}
%==============================================================================

\textbf{Taylor's Formula:}
\[
f(x) = \sum_{k=0}^{n} \frac{f^{(k)}(a)}{k!}(x-a)^k + R_n(x)
\]

\textbf{Remainder (Lagrange):} $\displaystyle R_n(x) = \frac{f^{(n+1)}(\xi)}{(n+1)!}(x-a)^{n+1}$ for some $\xi$ between $a$ and $x$.

\subsection{Maclaurin Series ($a = 0$)}
\begin{center}
\begin{tabular}{ll}
\toprule
\textbf{Function} & \textbf{Expansion} \\
\midrule
$e^x$ & $1 + x + \frac{x^2}{2!} + \frac{x^3}{3!} + \cdots$ \\[4pt]
$\ln(1+x)$ & $x - \frac{x^2}{2} + \frac{x^3}{3} - \cdots$ \\[4pt]
$\sin x$ & $x - \frac{x^3}{3!} + \frac{x^5}{5!} - \cdots$ \\[4pt]
$\cos x$ & $1 - \frac{x^2}{2!} + \frac{x^4}{4!} - \cdots$ \\[4pt]
$(1+x)^n$ & $1 + nx + \frac{n(n-1)}{2!}x^2 + \cdots$ \\
\bottomrule
\end{tabular}
\end{center}

\textbf{Linear:} $f(x) \approx f(a) + f'(a)(x-a)$ \qquad \textbf{Quadratic:} $f(x) \approx f(a) + f'(a)(x-a) + \frac{f''(a)}{2}(x-a)^2$

%==============================================================================
\section{Elasticities}
%==============================================================================

\textbf{Definition:}
\[
\varepsilon_f = \frac{x}{f(x)} \cdot f'(x) = \frac{d(\ln f)}{d(\ln x)}
\]

\textbf{Interpretation:} Percent change in $f$ per percent change in $x$.

\textbf{Rules:}
\begin{center}
\begin{tabular}{ll}
\toprule
\textbf{Operation} & \textbf{Elasticity Rule} \\
\midrule
Product: $f \cdot g$ & $\varepsilon_{f \cdot g} = \varepsilon_f + \varepsilon_g$ \\
Quotient: $f/g$ & $\varepsilon_{f/g} = \varepsilon_f - \varepsilon_g$ \\
Power: $f^n$ & $\varepsilon_{f^n} = n \cdot \varepsilon_f$ \\
\bottomrule
\end{tabular}
\end{center}

\textbf{Economics:} $|\varepsilon| > 1$: elastic; $|\varepsilon| < 1$: inelastic; $|\varepsilon| = 1$: unit elastic.

%==============================================================================
\section{Implicit Differentiation}
%==============================================================================

\textbf{Method:} Given $F(x, y) = 0$ where $y = y(x)$:
\begin{enumerate}
    \item Differentiate both sides with respect to $x$
    \item Apply chain rule: $\frac{d}{dx}[g(y)] = g'(y) \cdot \frac{dy}{dx}$
    \item Solve for $\frac{dy}{dx}$
\end{enumerate}

\textbf{Formula:} If $F(x, y) = 0$, then:
\[
\frac{dy}{dx} = -\frac{\partial F/\partial x}{\partial F/\partial y} = -\frac{F_x}{F_y}
\]

\textbf{Second Derivative:} Differentiate $\frac{dy}{dx}$ implicitly again, substituting the first derivative.

%==============================================================================
\section{Optimization (Single Variable)}
%==============================================================================

\textbf{Necessary condition:} $f'(x^*) = 0$ (critical point)

\textbf{Second Derivative Test:}
\begin{center}
\begin{tabular}{ll}
\toprule
\textbf{Condition} & \textbf{Conclusion} \\
\midrule
$f''(x^*) > 0$ & Local minimum \\
$f''(x^*) < 0$ & Local maximum \\
$f''(x^*) = 0$ & Inconclusive \\
\bottomrule
\end{tabular}
\end{center}

\textbf{Global Extrema on $[a,b]$:} (1) Find critical points in $(a,b)$, (2) Evaluate $f$ at critical points AND endpoints, (3) Compare values.

\textbf{Extreme Value Theorem:} If $f$ is continuous on $[a,b]$, then $f$ attains both a max and min on $[a,b]$.

%==============================================================================
\section{Multivariate Functions}
%==============================================================================

\textbf{Partial Derivatives:}
\[
f_x = \frac{\partial f}{\partial x} = \lim_{h \to 0} \frac{f(x+h, y) - f(x, y)}{h} \quad \text{(treat $y$ as constant)}
\]

\textbf{Second Partials:} $f_{xx}$, $f_{yy}$, $f_{xy}$, $f_{yx}$

\textbf{Schwarz's Theorem:} If $f_{xy}$ and $f_{yx}$ are continuous, then $f_{xy} = f_{yx}$.

\textbf{Gradient:} $\nabla f = \left(\frac{\partial f}{\partial x}, \frac{\partial f}{\partial y}\right)$ -- perpendicular to level curves.

\textbf{Level Curves:} Set of points where $f(x, y) = c$ (constant).

%==============================================================================
\section{Multivariate Optimization}
%==============================================================================

\textbf{First-Order Conditions:} $f_x(x^*, y^*) = 0$ AND $f_y(x^*, y^*) = 0$

\textbf{Hessian Matrix:}
\[
H = \begin{pmatrix} f_{xx} & f_{xy} \\ f_{yx} & f_{yy} \end{pmatrix}
\]

\textbf{Determinant:} $D = \det(H) = f_{xx} \cdot f_{yy} - (f_{xy})^2$

\textbf{Second-Order Conditions:}
\begin{center}
\begin{tabular}{lll}
\toprule
$D$ & $f_{xx}$ & \textbf{Conclusion} \\
\midrule
$D > 0$ & $f_{xx} > 0$ & Local minimum \\
$D > 0$ & $f_{xx} < 0$ & Local maximum \\
$D < 0$ & any & Saddle point \\
$D = 0$ & any & Inconclusive \\
\bottomrule
\end{tabular}
\end{center}

%==============================================================================
\section{Lagrange Multipliers}
%==============================================================================

\textbf{Problem:} Optimize $f(x, y)$ subject to $g(x, y) = c$

\textbf{Lagrangian:}
\[
\mathcal{L}(x, y, \lambda) = f(x, y) - \lambda(g(x, y) - c)
\]

\textbf{First-Order Conditions:}
\begin{align*}
\frac{\partial \mathcal{L}}{\partial x} &= f_x - \lambda g_x = 0 \\
\frac{\partial \mathcal{L}}{\partial y} &= f_y - \lambda g_y = 0 \\
\frac{\partial \mathcal{L}}{\partial \lambda} &= -(g(x, y) - c) = 0
\end{align*}

\textbf{Interpretation of $\lambda$:} $\lambda = \frac{df^*}{dc}$ = marginal value of relaxing constraint (shadow price).

\begin{tipbox}
\textbf{Template:} (1) Set up Lagrangian, (2) Take partial derivatives and set = 0, (3) Solve system, (4) Check boundaries if domain bounded, (5) Verify max/min.
\end{tipbox}

%==============================================================================
\section{Integration}
%==============================================================================

\subsection{Basic Integrals}
\begin{center}
\begin{tabular}{ll|ll}
\toprule
\textbf{Function} & \textbf{Integral} & \textbf{Function} & \textbf{Integral} \\
\midrule
$x^n$ ($n \neq -1$) & $\frac{x^{n+1}}{n+1} + C$ & $e^x$ & $e^x + C$ \\[4pt]
$\frac{1}{x}$ & $\ln|x| + C$ & $a^x$ & $\frac{a^x}{\ln a} + C$ \\[4pt]
$\sin x$ & $-\cos x + C$ & $\cos x$ & $\sin x + C$ \\
\bottomrule
\end{tabular}
\end{center}

\textbf{Fundamental Theorem:} $\displaystyle\int_a^b f(x)\,dx = F(b) - F(a)$ where $F'(x) = f(x)$

\subsection{Integration Techniques}

\textbf{By Parts:} $\displaystyle\int u\,dv = uv - \int v\,du$

\quad LIATE rule for choosing $u$: Logs, Inverse trig, Algebraic, Trig, Exponential

\textbf{Substitution:} $\displaystyle\int f(g(x)) \cdot g'(x)\,dx = \int f(u)\,du$ where $u = g(x)$

\textbf{Rational Functions:} (1) Long division if $\deg(P) \geq \deg(Q)$, (2) Partial fractions, (3) Integrate terms.

%==============================================================================
\section*{Quick Reference: Economic Applications}
%==============================================================================

\begin{center}
\begin{tabular}{ll}
\toprule
\textbf{Concept} & \textbf{Formula/Description} \\
\midrule
Cost function $C(q)$ & Typically increasing, often convex \\
Revenue $R(q)$ & $R = p \cdot q$ \\
Profit $\Pi(q)$ & $\Pi = R(q) - C(q)$ \\
Profit max condition & $MR = MC$ (marginal revenue = marginal cost) \\
Utility max condition & $\frac{MU_x}{MU_y} = \frac{p_x}{p_y}$ (MRS = price ratio) \\
Cobb-Douglas $Q = AL^\alpha K^\beta$ & CRS: $\alpha+\beta=1$; IRS: $\alpha+\beta>1$; DRS: $\alpha+\beta<1$ \\
\bottomrule
\end{tabular}
\end{center}

\end{document}