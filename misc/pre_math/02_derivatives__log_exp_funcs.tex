\documentclass[aspectratio=169]{beamer}
\usepackage[utf8]{inputenc}
\usepackage{amsmath, amssymb, amsthm}
\usepackage{tikz}
\usepackage{pgfplots}
\pgfplotsset{compat=1.17}

\usetheme{Madrid}
\usecolortheme{default}

\title{Derivatives: Rules and Applications}
\author{Mathematics Lecture Notes}
\date{}

% Define theorem style for beamer
\setbeamertemplate{theorems}[numbered]

\begin{document}

\frame{\titlepage}

\section{Introduction}

The derivative measures the rate of change of a function. In this lecture, we'll cover the fundamental rules for computing derivatives and explore derivatives of exponential and logarithmic functions.

\subsection{Definition of the Derivative}

For a function $f(x)$, the derivative at point $x$ is:
$$f'(x) = \lim_{h \to 0} \frac{f(x+h) - f(x)}{h}$$

\section{Properties of Exponents and Logarithms}

Before diving into derivatives, let's review the essential properties of exponentials and logarithms that we'll use throughout this lecture.

\subsection{Exponent Properties}

For any real numbers $a, b > 0$ and any real numbers $m, n$:

\begin{enumerate}
\item \textbf{Product rule:} $a^m \cdot a^n = a^{m+n}$

Example: $2^3 \cdot 2^5 = 2^8 = 256$

\item \textbf{Quotient rule:} $\frac{a^m}{a^n} = a^{m-n}$

Example: $\frac{x^5}{x^2} = x^{5-2} = x^3$

\item \textbf{Power of a power:} $(a^m)^n = a^{mn}$

Example: $(x^2)^3 = x^6$

\item \textbf{Power of a product:} $(ab)^n = a^n b^n$

Example: $(2x)^3 = 2^3 x^3 = 8x^3$

\item \textbf{Power of a quotient:} $\left(\frac{a}{b}\right)^n = \frac{a^n}{b^n}$

Example: $\left(\frac{x}{y}\right)^2 = \frac{x^2}{y^2}$

\item \textbf{Negative exponent:} $a^{-n} = \frac{1}{a^n}$

Example: $x^{-2} = \frac{1}{x^2}$

\item \textbf{Zero exponent:} $a^0 = 1$ (for $a \neq 0$)

\item \textbf{Fractional exponent:} $a^{1/n} = \sqrt[n]{a}$ and $a^{m/n} = \sqrt[n]{a^m} = (\sqrt[n]{a})^m$

Example: $8^{2/3} = (\sqrt[3]{8})^2 = 2^2 = 4$
\end{enumerate}

\subsection{Logarithm Properties}

The logarithm $\log_a x$ is the inverse of the exponential function: if $y = a^x$, then $x = \log_a y$.

The natural logarithm uses base $e$: $\ln x = \log_e x$.

\textbf{Key Identity:} $a^{\log_a x} = x$ and $\log_a(a^x) = x$

\textbf{Fundamental Properties:}

For $a, b > 0$, $a \neq 1$, and $x, y > 0$:

\begin{enumerate}
\item \textbf{Product rule:} $\log_a(xy) = \log_a x + \log_a y$

Example: $\ln(5 \cdot 3) = \ln 5 + \ln 3$

This is why logarithms are useful: they turn multiplication into addition!

\item \textbf{Quotient rule:} $\log_a\left(\frac{x}{y}\right) = \log_a x - \log_a y$

Example: $\log_2\left(\frac{8}{4}\right) = \log_2 8 - \log_2 4 = 3 - 2 = 1$

\item \textbf{Power rule:} $\log_a(x^r) = r \log_a x$

Example: $\ln(x^3) = 3\ln x$

This property is particularly useful in calculus!

\item \textbf{Change of base formula:} $\log_a x = \frac{\log_b x}{\log_b a} = \frac{\ln x}{\ln a}$

Example: $\log_2 10 = \frac{\ln 10}{\ln 2} \approx \frac{2.303}{0.693} \approx 3.32$

\item \textbf{Special values:}
\begin{itemize}
\item $\log_a 1 = 0$ (because $a^0 = 1$)
\item $\log_a a = 1$ (because $a^1 = a$)
\item $\ln e = 1$
\end{itemize}
\end{enumerate}

\subsection{Common Simplifications}

\textbf{Example 1:} Simplify $\ln(e^{3x})$
$$\ln(e^{3x}) = 3x \ln e = 3x \cdot 1 = 3x$$

\textbf{Example 2:} Simplify $e^{\ln(x^2)}$
$$e^{\ln(x^2)} = x^2$$

\textbf{Example 3:} Expand $\ln\left(\frac{x^2 \sqrt{y}}{z^3}\right)$
\begin{align*}
\ln\left(\frac{x^2 \sqrt{y}}{z^3}\right) &= \ln(x^2 \sqrt{y}) - \ln(z^3)\\
&= \ln(x^2) + \ln(\sqrt{y}) - \ln(z^3)\\
&= 2\ln x + \frac{1}{2}\ln y - 3\ln z
\end{align*}

\textbf{Example 4:} Condense $3\ln x - 2\ln y + \ln z$
$$3\ln x - 2\ln y + \ln z = \ln(x^3) - \ln(y^2) + \ln z = \ln\left(\frac{x^3 z}{y^2}\right)$$

\section{Differentiation Rules}

\subsection{Sum Rule}

The derivative of a sum is the sum of the derivatives.

\begin{theorem}[Sum Rule]
If $f(x)$ and $g(x)$ are differentiable, then:
$$(f + g)' = f' + g'$$
Or more explicitly:
$$\frac{d}{dx}[f(x) + g(x)] = \frac{d}{dx}f(x) + \frac{d}{dx}g(x)$$
\end{theorem}

\textbf{Example:}
$$\frac{d}{dx}(x^3 + 5x^2) = \frac{d}{dx}(x^3) + \frac{d}{dx}(5x^2) = 3x^2 + 10x$$

\textbf{Extension:} This extends to any finite number of functions:
$$\frac{d}{dx}[f_1(x) + f_2(x) + \cdots + f_n(x)] = f_1'(x) + f_2'(x) + \cdots + f_n'(x)$$

\subsection{Product Rule}

The derivative of a product is NOT simply the product of derivatives.

\begin{theorem}[Product Rule]
If $f(x)$ and $g(x)$ are differentiable, then:
$$(f \cdot g)' = f' \cdot g + f \cdot g'$$
Or:
$$\frac{d}{dx}[f(x) \cdot g(x)] = f'(x) \cdot g(x) + f(x) \cdot g'(x)$$
\end{theorem}

\textbf{Mnemonic:} ``First times derivative of second, plus second times derivative of first.''

\textbf{Example 1:}
$$\frac{d}{dx}(x^2 \sin x) = 2x \cdot \sin x + x^2 \cdot \cos x$$

\textbf{Example 2:}
$$\frac{d}{dx}[(3x+1)(x^2-2)] = 3(x^2-2) + (3x+1)(2x) = 3x^2 - 6 + 6x^2 + 2x = 9x^2 + 2x - 6$$

\subsection{Quotient Rule}

For the derivative of a quotient, we use the quotient rule.

\begin{theorem}[Quotient Rule]
If $f(x)$ and $g(x)$ are differentiable and $g(x) \neq 0$, then:
$$\left(\frac{f}{g}\right)' = \frac{f' \cdot g - f \cdot g'}{g^2}$$
Or:
$$\frac{d}{dx}\left[\frac{f(x)}{g(x)}\right] = \frac{f'(x) \cdot g(x) - f(x) \cdot g'(x)}{[g(x)]^2}$$
\end{theorem}

\textbf{Mnemonic:} ``Low dee-high minus high dee-low, over low-low'' (denominator times derivative of numerator minus numerator times derivative of denominator, all over denominator squared).

\textbf{Example 1:}
$$\frac{d}{dx}\left(\frac{x^2}{x+1}\right) = \frac{2x(x+1) - x^2(1)}{(x+1)^2} = \frac{2x^2 + 2x - x^2}{(x+1)^2} = \frac{x^2 + 2x}{(x+1)^2}$$

\textbf{Example 2:}
$$\frac{d}{dx}\left(\frac{\sin x}{x}\right) = \frac{x \cos x - \sin x \cdot 1}{x^2} = \frac{x \cos x - \sin x}{x^2}$$

\subsection{Chain Rule}

The chain rule is used to differentiate composite functions.

\begin{theorem}[Chain Rule]
If $y = f(g(x))$, then:
$$\frac{dy}{dx} = f'(g(x)) \cdot g'(x)$$
Or using Leibniz notation: if $y = f(u)$ and $u = g(x)$, then:
$$\frac{dy}{dx} = \frac{dy}{du} \cdot \frac{du}{dx}$$
\end{theorem}

\textbf{Intuition:} The rate of change of $y$ with respect to $x$ equals the rate of change of $y$ with respect to $u$ times the rate of change of $u$ with respect to $x$.

\textbf{Example 1:} Find $\frac{d}{dx}(3x+1)^5$

Let $u = 3x+1$, then $y = u^5$
$$\frac{dy}{dx} = \frac{dy}{du} \cdot \frac{du}{dx} = 5u^4 \cdot 3 = 15(3x+1)^4$$

\textbf{Example 2:} Find $\frac{d}{dx}\sin(x^2)$
$$\frac{d}{dx}\sin(x^2) = \cos(x^2) \cdot 2x = 2x\cos(x^2)$$

\textbf{Example 3:} Find $\frac{d}{dx}e^{x^3+2x}$
$$\frac{d}{dx}e^{x^3+2x} = e^{x^3+2x} \cdot (3x^2+2)$$

\subsection{Combining Rules}

Often, we need to combine multiple rules.

\textbf{Example:} Find $\frac{d}{dx}\left[\frac{(x^2+1)^3}{x}\right]$

Using quotient rule combined with chain rule:
\begin{align*}
\frac{d}{dx}\left[\frac{(x^2+1)^3}{x}\right] &= \frac{x \cdot 3(x^2+1)^2 \cdot 2x - (x^2+1)^3 \cdot 1}{x^2}\\
&= \frac{6x^2(x^2+1)^2 - (x^2+1)^3}{x^2}\\
&= \frac{(x^2+1)^2[6x^2 - (x^2+1)]}{x^2}\\
&= \frac{(x^2+1)^2(5x^2-1)}{x^2}
\end{align*}

\section{Higher-Order Derivatives}

The derivative of a derivative is called a higher-order derivative.

\subsection{Notation}

\begin{itemize}
\item \textbf{Second derivative:} $f''(x)$, $\frac{d^2f}{dx^2}$, or $\frac{d^2y}{dx^2}$
\item \textbf{Third derivative:} $f'''(x)$ or $\frac{d^3f}{dx^3}$
\item \textbf{$n$-th derivative:} $f^{(n)}(x)$ or $\frac{d^nf}{dx^n}$
\end{itemize}

\subsection{Examples}

\textbf{Example 1:} Find all derivatives of $f(x) = x^4 - 3x^3 + 2x - 5$
\begin{align*}
f'(x) &= 4x^3 - 9x^2 + 2\\
f''(x) &= 12x^2 - 18x\\
f'''(x) &= 24x - 18\\
f^{(4)}(x) &= 24\\
f^{(5)}(x) &= 0
\end{align*}

All subsequent derivatives are zero.

\textbf{Example 2:} Find $f''(x)$ for $f(x) = \sin x$
\begin{align*}
f'(x) &= \cos x\\
f''(x) &= -\sin x
\end{align*}

\textbf{Interpretation:}
\begin{itemize}
\item $f'(x)$: instantaneous rate of change (velocity if $f$ is position)
\item $f''(x)$: rate of change of the rate of change (acceleration if $f$ is position)
\item $f'''(x)$: jerk (rate of change of acceleration)
\end{itemize}

\section{Exponential Functions}

\subsection{The Natural Exponential Function}

The exponential function $e^x$ has a remarkable property: it is its own derivative.

\begin{theorem}
$$\frac{d}{dx}(e^x) = e^x$$
\end{theorem}

This is why $e$ is called the ``natural'' base for exponential functions.

\textbf{Examples:}

1. $\frac{d}{dx}(e^x + x^2) = e^x + 2x$

2. Using the chain rule:
$$\frac{d}{dx}(e^{3x}) = e^{3x} \cdot 3 = 3e^{3x}$$

3. Using the product rule:
$$\frac{d}{dx}(xe^x) = 1 \cdot e^x + x \cdot e^x = e^x(1+x)$$

4. Using the chain rule:
$$\frac{d}{dx}(e^{x^2+1}) = e^{x^2+1} \cdot 2x = 2xe^{x^2+1}$$

\subsection{General Exponential Functions}

For any base $a > 0$, $a \neq 1$:

\begin{theorem}
$$\frac{d}{dx}(a^x) = a^x \ln a$$
\end{theorem}

\textbf{Note:} When $a = e$, we get $\ln e = 1$, so $\frac{d}{dx}(e^x) = e^x \cdot 1 = e^x$.

\textbf{Examples:}

1. $\frac{d}{dx}(2^x) = 2^x \ln 2$

2. $\frac{d}{dx}(10^x) = 10^x \ln 10$

3. With chain rule:
$$\frac{d}{dx}(3^{2x}) = 3^{2x} \ln 3 \cdot 2 = 2\ln 3 \cdot 3^{2x}$$

\section{Logarithmic Functions}

\subsection{The Natural Logarithm}

The natural logarithm $\ln x$ is the inverse of $e^x$.

\begin{theorem}
$$\frac{d}{dx}(\ln x) = \frac{1}{x}, \quad x > 0$$
\end{theorem}

\textbf{Examples:}

1. $\frac{d}{dx}(\ln x + x^3) = \frac{1}{x} + 3x^2$

2. Using the chain rule:
$$\frac{d}{dx}[\ln(x^2+1)] = \frac{1}{x^2+1} \cdot 2x = \frac{2x}{x^2+1}$$

3. Using the chain rule:
$$\frac{d}{dx}[\ln(3x)] = \frac{1}{3x} \cdot 3 = \frac{1}{x}$$

Note: $\ln(3x) = \ln 3 + \ln x$, so the derivative is indeed $\frac{1}{x}$.

4. Product rule with logarithm:
$$\frac{d}{dx}[x \ln x] = 1 \cdot \ln x + x \cdot \frac{1}{x} = \ln x + 1$$

\subsection{General Logarithms}

For logarithm with base $a > 0$, $a \neq 1$:

\begin{theorem}
$$\frac{d}{dx}(\log_a x) = \frac{1}{x \ln a}, \quad x > 0$$
\end{theorem}

\textbf{Note:} When $a = e$, we get $\ln e = 1$, so $\frac{d}{dx}(\ln x) = \frac{1}{x}$.

\textbf{Examples:}

1. $\frac{d}{dx}(\log_{10} x) = \frac{1}{x \ln 10}$

2. $\frac{d}{dx}(\log_2 x) = \frac{1}{x \ln 2}$

\subsection{Logarithmic Differentiation}

For complicated products, quotients, or powers, logarithmic differentiation can simplify calculations.

\textbf{Method:}
\begin{enumerate}
\item Take $\ln$ of both sides
\item Differentiate implicitly
\item Solve for $\frac{dy}{dx}$
\end{enumerate}

\textbf{Example:} Find $\frac{dy}{dx}$ if $y = x^x$

\begin{align*}
\ln y &= \ln(x^x) = x \ln x\\
\frac{1}{y} \cdot \frac{dy}{dx} &= 1 \cdot \ln x + x \cdot \frac{1}{x} = \ln x + 1\\
\frac{dy}{dx} &= y(\ln x + 1) = x^x(\ln x + 1)
\end{align*}

\section{Common Derivatives Reference}

\subsection{Power Functions}
\begin{align*}
\frac{d}{dx}(c) &= 0 \quad \text{(constant)}\\
\frac{d}{dx}(x) &= 1\\
\frac{d}{dx}(x^n) &= nx^{n-1}\\
\frac{d}{dx}(\sqrt{x}) &= \frac{1}{2\sqrt{x}}\\
\frac{d}{dx}\left(\frac{1}{x}\right) &= -\frac{1}{x^2}
\end{align*}

\subsection{Exponential and Logarithmic Functions}
\begin{align*}
\frac{d}{dx}(e^x) &= e^x\\
\frac{d}{dx}(a^x) &= a^x \ln a\\
\frac{d}{dx}(\ln x) &= \frac{1}{x}\\
\frac{d}{dx}(\log_a x) &= \frac{1}{x \ln a}
\end{align*}

\subsection{Trigonometric Functions}
\begin{align*}
\frac{d}{dx}(\sin x) &= \cos x\\
\frac{d}{dx}(\cos x) &= -\sin x\\
\frac{d}{dx}(\tan x) &= \sec^2 x\\
\frac{d}{dx}(\cot x) &= -\csc^2 x\\
\frac{d}{dx}(\sec x) &= \sec x \tan x\\
\frac{d}{dx}(\csc x) &= -\csc x \cot x
\end{align*}

\subsection{Inverse Trigonometric Functions}
\begin{align*}
\frac{d}{dx}(\arcsin x) &= \frac{1}{\sqrt{1-x^2}}\\
\frac{d}{dx}(\arccos x) &= -\frac{1}{\sqrt{1-x^2}}\\
\frac{d}{dx}(\arctan x) &= \frac{1}{1+x^2}
\end{align*}

\section{Practice Problems}

\subsection{Problem Set}

\textbf{1.} Find $\frac{d}{dx}(x^3 + 5x^2 - 3x + 7)$

\textbf{2.} Find $\frac{d}{dx}[(2x+1)(x^2-3)]$

\textbf{3.} Find $\frac{d}{dx}\left(\frac{x^2+1}{x-2}\right)$

\textbf{4.} Find $\frac{d}{dx}[(x^2+3x)^{10}]$

\textbf{5.} Find $\frac{d}{dx}[e^{x^2}]$

\textbf{6.} Find $\frac{d}{dx}[\ln(x^3+2x)]$

\textbf{7.} Find $\frac{d^2}{dx^2}[x^4 - 2x^3 + x]$

\textbf{8.} Find $\frac{d}{dx}[x^2 e^x]$

\subsection{Solutions}

\textbf{1.} $3x^2 + 10x - 3$

\textbf{2.} $2(x^2-3) + (2x+1)(2x) = 2x^2 - 6 + 4x^2 + 2x = 6x^2 + 2x - 6$

\textbf{3.} $\frac{2x(x-2) - (x^2+1)(1)}{(x-2)^2} = \frac{2x^2 - 4x - x^2 - 1}{(x-2)^2} = \frac{x^2 - 4x - 1}{(x-2)^2}$

\textbf{4.} $10(x^2+3x)^9 \cdot (2x+3)$

\textbf{5.} $e^{x^2} \cdot 2x = 2xe^{x^2}$

\textbf{6.} $\frac{1}{x^3+2x} \cdot (3x^2+2) = \frac{3x^2+2}{x^3+2x}$

\textbf{7.} First derivative: $4x^3 - 6x^2 + 1$; Second derivative: $12x^2 - 12x$

\textbf{8.} $2x \cdot e^x + x^2 \cdot e^x = e^x(2x + x^2) = xe^x(2 + x)$

\section{Summary}

\begin{itemize}
\item \textbf{Sum Rule:} $(f+g)' = f' + g'$
\item \textbf{Product Rule:} $(fg)' = f'g + fg'$
\item \textbf{Quotient Rule:} $\left(\frac{f}{g}\right)' = \frac{f'g - fg'}{g^2}$
\item \textbf{Chain Rule:} $(f \circ g)' = f'(g(x)) \cdot g'(x)$
\item \textbf{Exponential:} $(e^x)' = e^x$ and $(a^x)' = a^x \ln a$
\item \textbf{Logarithm:} $(\ln x)' = \frac{1}{x}$ and $(\log_a x)' = \frac{1}{x \ln a}$
\item Higher-order derivatives represent rates of change of rates of change
\end{itemize}

These rules form the foundation for differential calculus and are essential for optimization, physics, and machine learning applications.

\end{document}
